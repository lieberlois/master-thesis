\section{Identity and Access Management}
\label{section:iam}
    Similar to how having a uniform strategy for observability across all environments is required, it is also essential to have such a strategy for identity and access management in all tooling in order to be able to maintain an IIoT system at scale. Since the proposed architecture employs GitOps (see \autoref{section:gitops}) as the continuous delivery strategy, the already existing role-based access control (RBAC) mechanisms of the Git repository hosting service (e.g. GitHub or GitLab) can be used for both code access and continuous delivery. In a scenario where the ``everything as code'' approach is fully implemented, those developers granted the necessary Git permissions are able to alter the configurations across the entire system. By having one GitOps controller per environment, or even per team per environment, the segregation of teams can easily be realized. Kubernetes offers a rich set of RBAC functionality with service accounts, roles and role bindings that can be managed through manifests in Git. By employing a multicluster management solution (see \autoref{section:multicluster-mgmt}) the RBAC configuration for access to the Kubernetes instances can be simplified and centralized across all clusters. 

    Lastly, access to applications in the system like observability tools can be unified by relying on a common standard like OAuth 2.0 and OpenID Connect. This allows to use a central, preferably already existing, identity provider like Azure Entra ID or the open-source project Keycloak for authentication and authorization across the whole system. 
