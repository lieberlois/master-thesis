\section{Requirements for the Hybrid Cloud IIoT Reference Architecture}
    Before introducing the hybrid cloud IIoT reference architecture, we must first think about the requirements, that the reference architecture must satisfy in order for it to be useful. Since the architecture in question is mainly focused on the actual implementation of such a system, we will concentrate on identifying and fulfilling common use cases whereas the architectures discussed in \autoref{subsection:high-level-ref} rather focused on requirements engineering and planning. 

    A reference architecture fundamentally has a set of domain-independent requirements, providing a standardized framework applicable across various contexts. It provides a common vocabulary for all stakeholders to base their discussions on, reusable designs and building blocks and industry best practices that are used as a constraint for designing more concrete architectures. Unlike solution architectures, they are designed to function as a blueprint that can be applied to multiple projects within a specific domain, whereas solution architectures are tailored to address the unique needs of a single project, focusing on specific requirements and solutions. Leveraging reference architectures as foundational templates for creating custom solution architectures can significantly lower development costs and shorten project timelines \cite{reference_architecture_in_general}.

    Focusing more on the IIoT domain, many more requirements arise and are mostly dependent on the project. Concerns like edge computing, low-latency workloads, high availability \& scalability, security \& compliance, heterogeneous protocols and many more typically have to be addressed and must be dealt with in an IIoT reference architecture. We discovered in \autoref{subsection:automation-pyramid} that systems based on the automation pyramid which is seen as a standard in the world of operational technology have fundamental shortcomings like stifling of innovation and high cost due to point-to-point integrations between systems in the layered design, a new reference architecture must solve these known issues. High-level reference architectures discussed in \autoref{subsection:high-level-ref} provide a good basis for discussion of an overall design, but lack specificity when it comes to the actual implementation of a real system. Lastly, more concrete reference architectures designed by cloud providers as discussed in \autoref{subsection:cloud-ref} often appear to lack flexibility and often kill important use cases like edge computing, which the reference architecture focused in this work must consider. It becomes evident that there is a notable gap in reference architectures on the market that offer a good balance between those that mainly serve as a high-level discussion and design basis and those that are concrete enough for implementation but often too specific and hence not suitable for many projects. The reference architecture that will be introduced in the next chapters aims to fill that gap.

