\section{Multicluster Management}
\label{section:multicluster-mgmt}

    While having good observability (see \autoref{section:observability}) certainly helps with monitoring and understanding a large-scale IIoT system, the management of the high amount of environments still poses a challenge. It is desirable to be able to manage each system from a central location, especially in critical topics like upgrading Kubernetes versions, configuring role-based access, single sign-on and more. Since all environments are based on Kubernetes a uniform solution can be used for this, which is where ``Multicluster Management'' solutions come in handy. Since this component is mostly just operating on top of the infrastructure, this work will only briefly mention some common solutions rather than diving deeper into the topic.\newline

    One of the most common open-source multicluster management solutions in the Kubernetes environment is Rancher. The developers describe Rancher as ``a complete software stack for teams adopting containers. It addresses the operational and security challenges of managing multiple Kubernetes clusters while providing DevOps teams with integrated tools for running containerized workloads'' \cite{rancher_platform}. Another similar enterprise product is ``D2iQ Kubernetes Management Platform - DKP'' which follows a common goal. Overall, the solutions allow to centrally operate many Kubernetes clusters by e.g.\ providing tooling to configure role-based access management, network access and cost analysis across all managed clusters. This is typically achieved by running a control plane on a dedicated management cluster, that is only used for the management of other so-called workload clusters. Each workload cluster runs an agent which creates a tunnel back to the control plane, often involving a proxy so that all clusters can be accessed via the management cluster. This is especially interesting with restricted networks like in cloud-to-edge scenarios since tunnels are securely encrypted and thus allow for safe access \cite{building_iiot}. Many multicluster management solutions like Rancher also have features that allow for provisioning clusters onto different infrastructure ranging from cloud providers to bare-metal. Unfortunately, the multicluster management platforms in their current state often lack extensibility, which makes use cases like provisioning Kubernetes onto own bare-metal hardware difficult if not infeasible.

    Another tool for multicluster management is ``ClusterAPI'' which has emerged as an increasingly prominent project in the domain of cloud computing and orchestration \cite{efficient_k8s_capi}. It mainly focuses on simplifying the management of the full lifetime of workload clusters including provisioning, operating and upgrading by abstracting the underlying complexity. ClusterAPI will be further discussed in \autoref{subsection:capi}.\newline

    \noindent In summary, a multicluster management solution is helpful with a growing amount of environments and helps to manage each environment by providing a central control plane that facilitates oversight, coordination, and operation across the variety of environments. This enhances efficiency and consistency regarding policy enforcement across all environments and thus helps to satisfy the requirement of central management across an entire IIoT system.
