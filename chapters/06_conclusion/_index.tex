\chapter{Conclusion}
\label{chapter:conclusion}
        In this work, we found a significant gap in existing standards and reference architectures for modern Industrial Internet of Things (IIoT) systems, particularly in their ability to meet evolving requirements. In \autoref{chapter:theoretical-background}, we discovered that traditional and often OT-driven models, such as the automation pyramid, face challenges in terms of scalability and innovation due to point-to-point integrations. Furthermore, while high-level frameworks like RAMI 4.0 and IIRA offer valuable concepts, they fall short in practical implementation guidance. Additionally, architectures from major cloud vendors often lead to a vendor lock-in and limit functionalities like edge computing. This discovery underlines the necessity for new approaches in designing scalable, efficient, and secure IIoT systems that align with current technological methods and industry needs. \newline
    
        We then introduced a new modern reference architecture encompassing a three-tier architecture and the concept of the unified namespace in \autoref{chapter:architecture-proposal}. We showed how having an edge, fog and cloud environment with one pub-sub broker per fog environment and a central pub-sub broker in the cloud can provide the means for building a scalable IIoT platform that satisfies modern requirements for such systems. To be able to realize a system of such scale successfully, we discussed GitOps, observability, identity and access management, orchestration and multicluster management. \newline
    
        In \autoref{chapter:infrastructure-provisioning}, we discussed the topic of provisioning the necessary infrastructure, particularly Kubernetes, for the reference architecture. It became apparent that the Kubernetes-native project ``ClusterAPI'' can be a solid technology choice since it is capable of provisioning Kubernetes on a large variety of infrastructure ranging from physical machines to virtual machines and even cloud providers while offering a uniform interface to the developers. We also examined how using virtualization tools like VMware vSphere can both simplify and robustify the process of provisioning infrastructure at the cost of a small loss in performance. \newline

        Finally in \autoref{chapter:poc}, we implemented the reference architecture in a proof-of-concept.  We saw that the strategy for realizing this architecture in the real world, which was developed in this thesis, works effectively, demonstrating the practical feasibility and scalability of our approach. The successful implementation of the architecture across all environments, including the setup of a unified namespace based on MQTT using the HiveMQ broker, together with the integration of the key technologies Kubernetes, ClusterAPI, and FluxCD as the GitOps controller resulted in a flexible and robust setup. This proof-of-concept not only confirms the viability of the proposed architecture and implementation strategy but also provides a basis for the future development of IIoT systems, allowing for further innovation and efficiency improvements. \newline
    

    \section{Further Work}
    \label{section:further-work}

        While the research and findings presented in this thesis lay the groundwork for building modern and scalable IIoT systems, this work leaves some points open for future work on this topic. One important aspect is the necessity of an in-depth security strategy for IIoT systems. While we had the topic of security in mind at all times, this thesis only scratches the surface of the complex and evolving landscape of cybersecurity. Since this can be seen as one of the most important requirements for an IIoT system in the real world, this is a crucial topic for further research. Another topic is a strategy for the development of domain services operating on the IIoT platform described in this work. The presented strategy mainly deals with infrastructure and leaves designing a strategy for creating workloads within the IIoT system open for future investigation. Lastly, we discovered a lack of support for GitOps in the multicluster management platform Rancher. Subsequent research could solve this by developing custom Kubernetes operators that transform the manual onboarding process of Kubernetes clusters into the management platform into a fully declarative, GitOps-driven approach or by contributing to the open-source Rancher project. 
    
    \section{Acknowledgements}

        First and foremost, I would like to thank Prof. Dr. Bernhard Bauer and Prof. Dr. Alexander Knapp from the University of Augsburg for supervising this work and assisting me throughout the whole duration. I also want to show my appreciation to Lars Gielsok who supervised this work from the side of the MaibornWolff GmbH and supported this work immensely.\newline

        Completing this thesis involved much more than mere supervision, and I would like to extend my gratitude to several individuals whose contributions were invaluable. Thanks to Marc Jäckle (MaibornWolff GmbH) and Sebastian Wöhrl (MaibornWolff GmbH) for sharing incredible amounts of knowledge, providing resources, assisting in the implementation and overall offering me to work on this topic for my master's thesis. Last but not least, I am also grateful to Martin Zehetmayer (MaibornWolff GmbH). Despite having no direct connection to this thesis, he went out of his way to offer essential guidance and support, particularly in the context of bare-metal server management, enriching both the theoretical and practical aspects of this thesis.

        