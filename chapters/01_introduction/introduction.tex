The Internet of Things (IoT) is a transformative force within the landscape of technology, especially in the industrial sector. A key component of this technological journey is industrial IoT (IIoT), which brings digitalization into industrial manufacturing. Due to the rise of IIoT, we can see a rapid increase both in the number of connected devices and the data being created by them.

\section{Motivation}
It is apparent that building scalable IIoT systems that are capable of handling the sheer velocity and volume of data is getting more and more difficult. Being able to connect devices from different manufacturers using a wide and heterogeneous variety of protocols poses another demanding challenge. Due to modern requirements like low latency workloads or large-scale stream analytics, we also see that we have to deal with a large number of different environments from edge devices in a factory to on-premises systems and up to cloud systems which results in highly distributed systems and brings even more complexity into an IIoT platform.

This thesis aims to address these challenges, focusing on the creation and implementation of effective solutions for IIoT systems, especially in the context of hybrid cloud scenarios. Due to the significant expansion of IIoT across the world, it becomes imperative to provide solutions for these hurdles. Since we see a lack of standards, architectures and guidelines on the market that confront these emerging issues, we want to discuss a new reference architecture that is capable of serving as a basis for the implementation of modern IIoT systems. Even though the reference architecture dealt with in this work is already present, the absence of a strategy for implementing this architecture in real-world industrial IoT scenarios remains a significant gap. Thus, this thesis also intends to close this gap by offering a concrete yet adaptable strategy on the path from the high-level reference architecture to a fully functional IIoT system.

\section{Goals}

The primary goal of this work is to provide a comprehensive framework consisting of a reference architecture and a strategy for implementing it in the real world, tailored to tackle the challenges in the development of industrial IoT systems. For this, we aim to analyze the landscape around IIoT and identify the main challenges such as edge computing, automation, orchestration, the lack of standards, architectures and guidelines, and the management of a large scale hybrid cloud setup, while providing guidance on how to solve them. By implementing a proof of concept for the infrastructure of the suggested architecture, we want to demonstrate the practical applicability of this strategy in a real project while introducing and contextualizing all relevant tools and concepts that are required. Furthermore, another goal of this thesis is to conduct a thorough analysis of the existing common strategies and architectures in the IIoT environment and to understand why they are no longer sufficient for current use cases. 


\section{Constraints}

While this thesis addresses several perspectives of IIoT platforms, we also need to delineate the boundaries of this work. To begin with, this work does not deal with a large variety of IoT in general but has a predominant focus on industrial IoT in hybrid cloud scenarios. Due to the emphasis on the infrastructure of IIoT systems, the actual implementation of domain services running on an IIoT platform is also beyond the scope of this thesis, and thus is a great opportunity for further research. Lastly, while both the reference architecture and the proof of concept implementation have security features in mind, this work does not perform a comprehensive analysis regarding a detailed security strategy for an IIoT platform.


\section{Structure}

The work begins with the theoretical background in \autoref{chapter:theoretical-background}, exploring the current landscape of industrial IoT. This includes discussions on GitOps, orchestration, and finally common patterns and architectures in the domain of IIoT. \autoref{chapter:architecture-proposal} shifts the focus to a proposal for a modern reference architecture, detailing its requirements and components. Here topics such as the unified namespace, the application of GitOps in IIoT, and system management strategies are introduced. In \autoref{chapter:infrastructure-provisioning} the topic of provisioning the infrastructure required for a large-scale IIoT system following the newly introduced reference architecture is addressed. The discussion revolves around topics like bare-metal infrastructure, Kubernetes, ClusterAPI, and virtualization. All of this can be seen in \autoref{chapter:poc}, where the proposed architecture was implemented in the form of a proof of concept project, displaying all of the components and techniques from the previous chapters. Finally, \autoref{chapter:conclusion} concludes this thesis by summarizing the findings and potential areas for future work, along with acknowledgments to those who supported this work.




