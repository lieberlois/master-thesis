\subsection{State of Reference Architectures}
    Overall we can see a shortage in reference architectures on the market, that are capable of satisfying the requirements of modern  IIoT systems. We discovered that older designs like the automation pyramid (\autoref{subsection:automation-pyramid}) mainly struggle with scalability and innovation due to point-to-point integrations. Regarding architectures, we saw that high-level reference architectures (\autoref{subsection:high-level-ref}) like RAMI 4.0 and the IIRA are not concrete enough for actual implementation, but still contain many ideas and concepts that are very relevant to this day and already solve many issues of systems strictly following the automation pyramid. Finally, we discovered that the reference architectures designed by the large cloud vendors (\autoref{subsection:cloud-ref}) like Microsoft Azure not only impose a vendor lock-in but are also very limiting and make common use cases like edge computing hard, if not impossible, to implement. The upcoming \autoref{chapter:architecture-proposal} will introduce a new, modern reference architecture that aims to solve this blank spot in the market.