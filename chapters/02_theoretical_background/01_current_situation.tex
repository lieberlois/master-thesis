\section{Current Situation}
\label{section:current-situation}
    To understand the rationale of this thesis one must first look at the current situation of industrial IoT (IIoT). With movements like Industry 4.0 and Smart Manufacturing, the pace of digitalization in industrial production is higher than ever. While typical IoT systems are often uncritical systems like smart home devices, the application in the manufacturing sector imposes the inevitable need to solve new challenges. Dealing with the combination of the high frequency and volume of data recorded by manufacturing devices and the fact that systems often require real-time functionality with low latency communication means careful system engineering beginning with the setup of hardware and networking. Due to devices being very constrained regarding their computational resources highly distributed systems have to be built, which brings a high level of complexity into the system. Also since machines in manufacturing are built by many different vendors or just run old software, a very heterogeneous set of communication protocols makes building a uniform and central platform a difficult task. Due to the fact that the actual manufacturing process often depends on the IIoT system today, requirements like security and availability are also becoming increasingly important. Since malfunctions, vulnerabilities or downtimes of the system could lead to a halt in production, protecting systems from attacks and implementing state-of-the-art high availability is essential to fulfill the requirements of modern IIoT systems. Because of the strongly increasing amount of smart devices in production and the growing scale of IIoT systems which often include multiple environments like a cloud environment, one on-premises environment per production site and many edge environments per site, a high degree of automation is crucial for success. Maintaining the systems manually is often simply not possible due to the sheer scale of the systems. Onboarding and provisioning of infrastructure as well as software and hardware maintenance are just two examples of procedures that are infeasible to perform manually at this scale. Note that these are just a few of the many requirements that lay out the challenges of designing, building and operating IIoT systems in the modern world \cite{fraunhofer_whitepaper}.\newline

    In the current state, we can see a shortage of standards and decision bases on the market, as will be further discussed in this work. While reference architectures that claim to tackle the typical challenges exist, many are not open source or are too generic to actually be helpful in real-world implementations. Others are often so tied to a particular vendor that they are unfeasible in many scenarios due to project-specific circumstances like existing hardware or cloud provider choice. There also exists a variety of architectures that work fine for the production but fall short in other essential requirements like scalability or availability. Many architectures also struggle with the conjoining of the worlds of operation technology (OT) and information technology (IT) due to the very different viewpoints and requirements both worlds impose on an IIoT system. While the OT-oriented architectures are mainly focused on keeping the manufacturing process running, they often stifle innovation in the IT world e.g.\ by building point-to-point integrations between systems rather than having all data and communication centralized for everyone to work with easily at low integration cost. Lastly, a variety of reference architectures, especially those of modern cloud providers, fail to satisfy requirements like edge/on-premises computing (\autoref{section:edge-computing}) or hybrid cloud (\autoref{section:hybrid-cloud-structure}) and thus quickly become unsatisfactory for many projects. 

    For these reasons, the company ``MaibornWolff GmbH'' created a reference architecture for hybrid cloud IIoT systems (\autoref{chapter:architecture-proposal}), that aims to satisfy the requirements and solve the challenges mentioned above. In this work, this reference architecture and all other practices and components necessary for it will be explained. Also, a strategy for realizing the architecture in the real world, in particular for setting up the required infrastructure, will be developed in this thesis and implemented in a proof of concept project, since the reference architecture only provides a target state without guidance on the actual implementation. Note that this work is conducted together with ``MaibornWolff GmbH''.

    In this thesis, the plain term ``reference architecture'' will from now on always refer to the reference architecture created by ``MaibornWolff GmbH'' if nothing further is specified.
    
    