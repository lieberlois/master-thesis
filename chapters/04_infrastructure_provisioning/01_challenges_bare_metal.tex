\section{Advantages and Challenges in Bare-Metal Infrastructure}

    In IIoT systems having own hardware is almost always a non-negotiable factor, especially when it comes to edge computing (see \autoref{section:edge-computing}), which is why thinking about bare-metal infrastructure is inevitable in IIoT systems. Before taking a look at solutions for bare-metal infrastructure provisioning, we must first look at why bare-metal is such a challenge in the first place. The layering of abstractions is a recurring strategy in the history of computing. However, building abstractions over bare-metal machines is still a central problem that needs to be addressed, since physical machines are the fundamental pieces that underlie everything in the software ecosystem. The need for properly configured and managed hardware that is solid and capable of fulfilling today's requirements regarding load, stress, strain and even external influences like fire, earthquakes or a pandemic is higher than ever. The main issue regarding building a solid abstraction over bare-metal nowadays is the lack of standardization in APIs for provisioning and management. APIs that exist, OpenStack being the most common one, are often extremely complex and impose huge amounts of operational effort \cite{building_future_on_metal}. Other than that, maintaining bare-metal infrastructure has several other drawbacks. These include the high effort and complexity, the necessity for dedicated specialists, higher initial cost and lower flexibility and scalability compared to cloud resources. Once the machines are provisioned, the lack of services managed by cloud providers like object storage, virtual networks or databases poses another challenge. \newline

    Using bare-metal infrastructure also brings several beneficial factors however. First of all, it makes use cases like edge computing and low-latency applications possible, which are common requirements in IIoT. Owning hardware can also be more cost-effective compared to utilizing cloud provider resources, especially after the initial investment (buying hardware, engineering cost for abstractions, etc.) is amortized. This is especially noticeable in performance efficiency, since the full computational power of the hardware is available, compared to cloud-managed machines which often experience overhead due to virtualization. Furthermore, bare-metal machines offer higher standards when it comes to security, isolation and compliance as there is more control over both physical and network access. Moreover, while resources are shared among multiple tenants in a cloud system, which can lead to fluctuations in performance (``noisy neighbor problem''), bare-metal machines provide dedicated resources and thus offer more predictable performance. Custom hardware also performs better for workloads that are not fit for virtualized environments like database servers or GPU-heavy workloads for machine learning. Lastly, having open solutions built upon own hardware results in no dependence on a cloud provider thus minimizing the risk of a vendor lock-in \cite{building_future_on_metal}.

    