\chapter*{Abstract}

In the age of digitalization, more and more devices are being connected to Internet of Things (IoT) platforms. It is estimated that the number of edge devices will grow immensely in the next years and that the market around edge computing will even overtake the market in the cloud. In industrial IoT (IIoT) scenarios, one of the main challenges is handling the enormous loads of data due to the high frequency of devices like sensors. While generating data on devices is simple, building large-scale IIoT platforms capable of handling such data is becoming increasingly difficult. Additionally, achieving full connectivity is a hard task due to the heterogeneity of devices and protocols. Another major challenge in building IIoT platforms is the high amount of environments spanning from edge devices across on-premise systems up to the cloud. Hybrid cloud platforms are necessary for several reasons in these scenarios. For example, edge devices may not be allowed to connect to the cloud due to security concerns, thus the need for an on-premises system. Additionally, latency-critical workloads cannot tolerate delays caused by networking into the cloud. 

This work will deal with strategies, techniques, and architectures to solve these challenges. The current situation around industrial IoT will be evaluated and the major problems will be identified. Domains like edge computing, automation and orchestration, common IIoT patterns and architectures, and bare-metal machine management will be explored. To successfully implement industrial IoT systems that are capable of fulfilling all of these requirements, building upon reference architectures is essential. This thesis will discuss the shortcomings of many existing reference architectures, and then introduce a new reference architecture that is based on the concept ``unified namespace'', that aims to act as a basis for building modern industrial IoT systems at scale. All concepts necessary to understand and implement the architecture will be explored.

After the theoretical part, a strategy for implementing the newly introduced reference architecture for IIoT platforms in hybrid cloud scenarios will be introduced and then implemented based on the theory discussed in the previous chapters, followed by an evaluation of the system built as a proof-of-concept project. A wide range of technologies and techniques that help build the complete stack of this IIoT platform will be introduced and contextualized. Alternatives will be evaluated, given that the IIoT use case comes with many domain-specific requirements like security demands that must be supported. Finally, a  high level of automation is essential for success, which will be the main focus of this work.